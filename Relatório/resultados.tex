\section{Resultados}

\subsection{KNN}

Visando obter-se as melhores acurácias e tempos de treinamento, selecionou-se o valor K = 51, obtendo-se os resultados apresentados na Tabela \ref{table:resultadosKNN}:

\begin{table}[h]
\centering
\caption{Resultados para o KNN sendo K = 51}
\vspace{0.2cm}
\begin{tabular}{c|c|c|c|c|c}
Partição & Acurácia & F-medida & Precisão & Revocação & Tempo \\
\hline
1  & 0.74607 & 0.74607 & 0.74607 & 0.74607 & 37.549 \\
2  & 0.75515 & 0.75515 & 0.75515 & 0.75515 & 37.233 \\
3  & 0.74925 & 0.74925 & 0.74925 & 0.74925 & 37.347 \\
4  & 0.75603 & 0.75603 & 0.75603 & 0.75603 & 38.548 \\
5  & 0.76179 & 0.76179 & 0.76179 & 0.76179 & 39.026 \\
6  & 0.74662 & 0.74662 & 0.74662 & 0.74662 & 38.081 \\
7  & 0.75293 & 0.75293 & 0.75293 & 0.75293 & 37.536 \\
8  & 0.75183 & 0.75183 & 0.75183 & 0.75183 & 36.817 \\
9  & 0.75803 & 0.75803 & 0.75803 & 0.75803 & 36.604 \\
10 & 0.74341 & 0.74341 & 0.74341 & 0.74341 & 36.824 \\
\hline
Média & 0.75211 & 0.75211 & 0.75211 & 0.75211 & 37.557

\end{tabular} 
\label{table:resultadosKNN}
\end{table}

\subsection{Regressão logística}

Visando obter-se as melhores acurácias e tempos de treinamento, selecionou-se a hipótese linear com um fator de regularização \(\lambda\) = 100, obtendo-se os resultados apresentados na Tabela \ref{table:resultadosRL}.

\begin{table}[h]
\centering
\caption{Resultados para a Regressão Logística sendo a hipótese linear e \(\lambda\) = 100}
\vspace{0.2cm}
\begin{tabular}{c|c|c|c|c|c}
Partição & Acurácia & F-medida & Precisão & Revocação & Tempo \\
\hline
1  & 0.86062 & 0.85909 & 0.86581 & 0.85754 & 56.846 \\      
2  & 0.86322 & 0.86180 & 0.86875 & 0.86040 & 30.724 \\      
3  & 0.85977 & 0.85811 & 0.86674 & 0.85663 & 54.428 \\      
4  & 0.86047 & 0.85899 & 0.86679 & 0.85769 & 43.907 \\      
5  & 0.86992 & 0.86865 & 0.87501 & 0.86724 & 43.519 \\      
6  & 0.86648 & 0.86497 & 0.87144 & 0.86328 & 33.003 \\      
7  & 0.86945 & 0.86790 & 0.87574 & 0.86622 & 51.874 \\    
8  & 0.86793 & 0.86640 & 0.87390 & 0.86475 & 41.304 \\      
9  & 0.86943 & 0.86804 & 0.87503 & 0.86652 & 43.230 \\      
10 & 0.86492 & 0.86317 & 0.87150 & 0.86134 & 30.503 \\
\hline
Média & 0.86522 & 0.86371 & 0.87107 & 0.86216 & 42.934

\end{tabular} 
\label{table:resultadosRL}
\end{table}

\subsection{Redes Neurais Artificiais}
	
	Visando obter-se as melhores acurácias e tempos de treinamento, selecionou-se como taxa de aprendizagem \(\lambda\) = 0,06 e quantidade de neurônios na camada intermediária com 50 elementos, obtendo-se os resultados apresentados na Tabela \ref{table:resultadosRNA}.
	
	
\begin{table}[h]
\centering
\caption{Resultados para RNA}
\vspace{0.2cm}
\begin{tabular}{c|c|c|c|c|c}
Partição & Acurácia & F-medida & Precisão & Revocação & Tempo \\
\hline
1  & 0.85583 & 0.85473 & 0.85783 & 0.85361 & 94.369 \\
2  & 0.85800 & 0.85710 & 0.85953 & 0.85617 & 102.94 \\
3  & 0.86704 & 0.86559 & 0.87189 & 0.86397 & 119.73 \\
4  & 0.86070 & 0.85964 & 0.86340 & 0.85850 & 90.938 \\
5  & 0.86618 & 0.86513 & 0.86927 & 0.86395 & 101.33 \\
6  & 0.86188 & 0.86067 & 0.86453 & 0.85935 & 105.95 \\
7  & 0.86869 & 0.86717 & 0.87472 & 0.86552 & 129.29 \\
8  & 0.86945 & 0.86829 & 0.87212 & 0.86692 & 148.67 \\
9  & 0.86756 & 0.86626 & 0.87233 & 0.86484 & 119.64 \\
10 & 0.86504 & 0.86379 & 0.86751 & 0.86239 & 113.08 \\
\hline
Média & 0.86404 & 0.86284 & 0.86731 & 0.86152 & 112.59 \\
\end{tabular} 
\label{table:resultadosRNA}
\end{table}

\subsection{Máquinas de vetores de suporte}

Visando obter-se as melhores acurácias e tempos de treinamento, selecinaram-se duas opções para o SVM, uma com kernel linear e parâmetro de custo \emph{C} = 0.01, apresentando-se os resultados na Tabela \ref{table:resultadosSVMLinear} e uma com kernel radial, parâmetros de custo \emph{C} =1 e parâmetro \(\gamma\) = 0.01, apresentando os resultados na Tabela \ref{table:resultadosSVMRadial}

\begin{table}[h]
\centering
\caption{Resultados para SVM com kernel linear e C = 0.01}
\vspace{0.2cm}
\begin{tabular}{c|c|c|c|c|c}
Partição & Acurácia & F-medida & Precisão & Revocação & Tempo \\
\hline
1  & 0.84725 & 0.84635 & 0.84835 & 0.84553 & 387.15 \\ 
2  & 0.85753 & 0.85665 & 0.85894 & 0.85575 & 371.86 \\
3  & 0.85985 & 0.85886 & 0.86134 & 0.85784 & 398.54 \\
4  & 0.85467 & 0.85379 & 0.85627 & 0.85291 & 396.04 \\
5  & 0.86017 & 0.85941 & 0.86129 & 0.85862 & 370.89 \\
6  & 0.85363 & 0.85271 & 0.85463 & 0.85186 & 371.20 \\
7  & 0.86295 & 0.86198 & 0.86457 & 0.86094 & 371.87 \\
8  & 0.86137 & 0.86044 & 0.86265 & 0.85948 & 373.39 \\
9  & 0.85854 & 0.85770 & 0.85991 & 0.85684 & 371.56 \\
10 & 0.85725 & 0.85617 & 0.85880 & 0.85508 & 371.22 \\
\hline
Média & 0.85732 & 0.85641 & 0.85867 & 0.85548 & 378.37 \\

\end{tabular} 
\label{table:resultadosSVMLinear}
\end{table}

\begin{table}[h]
\centering
\caption{Resultados para SVM com kernel radial, C = 1 e \(\gamma\) = 0.01}
\vspace{0.2cm}
\begin{tabular}{c|c|c|c|c|c}
Partição & Acurácia & F-medida & Precisão & Revocação & Tempo \\
\hline
1  & 0.84725 & 0.84635 & 0.84835 & 0.84553 & 387.15 \\ 
2  & 0.85753 & 0.85665 & 0.85894 & 0.85575 & 371.86 \\
3  & 0.85985 & 0.85886 & 0.86134 & 0.85784 & 398.54 \\
4  & 0.85467 & 0.85379 & 0.85627 & 0.85291 & 396.04 \\
5  & 0.86017 & 0.85941 & 0.86129 & 0.85862 & 370.89 \\
6  & 0.85363 & 0.85271 & 0.85463 & 0.85186 & 371.20 \\
7  & 0.86295 & 0.86198 & 0.86457 & 0.86094 & 371.87 \\
8  & 0.86137 & 0.86044 & 0.86265 & 0.85948 & 373.39 \\
9  & 0.85854 & 0.85770 & 0.85991 & 0.85684 & 371.56 \\
10 & 0.85725 & 0.85617 & 0.85880 & 0.85508 & 371.22 \\
\hline
Média & 0.85732 & 0.85641 & 0.85867 & 0.85548 & 378.37 \\

\end{tabular} 
\label{table:resultadosSVMRadial}
\end{table}

\subsection{Naive Bayes}

\begin{table}[h]
\centering
\caption{Resultados para Naive Bayes}
\vspace{0.2cm}
\begin{tabular}{c|c|c|c|c|c}
Partição & Acurácia & F-medida & Precisão & Revocação & Tempo \\
\hline
1  & 0.84255 & 0.84031 & 0.85452 & 0.83972 & 1.5254 \\
2  & 0.84731 & 0.84506 & 0.86158 & 0.84486 & 1.5465 \\
3  & 0.84693 & 0.84463 & 0.86029 & 0.84408 & 1.5446 \\
4  & 0.84498 & 0.84278 & 0.85891 & 0.84272 & 1.5119 \\
5  & 0.84862 & 0.84670 & 0.86086 & 0.84650 & 1.5206 \\
6  & 0.84920 & 0.84690 & 0.86174 & 0.84603 & 1.5198 \\
7  & 0.84499 & 0.84272 & 0.85898 & 0.84251 & 1.5126 \\
8  & 0.85428 & 0.85205 & 0.86754 & 0.85119 & 1.5395 \\
9  & 0.85367 & 0.85163 & 0.86632 & 0.85104 & 1.5342 \\
10 & 0.84597 & 0.84349 & 0.85954 & 0.84274 & 1.5229 \\
\hline
Média & 0.84785 & 0.84563 & 0.86103 & 0.84514 & 1.5278 \\

\end{tabular} 
\label{table:resultadosNB}
\end{table}

\section{Resultados gerais}

O comparativo geral entre os métodos é apresentado na Tabela \ref{table:resultadosGerais} pela média das execuções dos métodos, os melhores resultados para cada parâmetros esta destacado em negrito.

\begin{table}[h]
\centering
\caption{Resultados gerais}
\vspace{0.2cm}
\begin{tabular}{c|c|c|c|c|c}
Método & Acurácia & F-medida & Precisão & Revocação & Tempo \\
\hline
KNN                & 0.75211 & 0.75211 & 0.75211 & 0.75211 & 37.557 \\
Regresão Log. & \textbf{0.86522} & \textbf{0.86371} & \textbf{0.87107} & \textbf{0.86216} & 42.934 \\
RNA                & 0.86404 & 0.86284 & 0.86731 & 0.86152 & 112.59 \\
SVM Linear         & 0.85732 & 0.85641 & 0.85867 & 0.85548 & 378.37 \\
SVM Radial         & 0.85732 & 0.85641 & 0.85867 & 0.85548 & 378.37 \\
Naive Bayes        & 0.84785 & 0.84563 & 0.86103 & 0.84514 & \textbf{1.5278} \\
\end{tabular}
\label{table:resultadosGerais}
\end{table}


